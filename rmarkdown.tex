% Options for packages loaded elsewhere
\PassOptionsToPackage{unicode}{hyperref}
\PassOptionsToPackage{hyphens}{url}
%
\documentclass[
]{article}
\usepackage{amsmath,amssymb}
\usepackage{lmodern}
\usepackage{iftex}
\ifPDFTeX
  \usepackage[T1]{fontenc}
  \usepackage[utf8]{inputenc}
  \usepackage{textcomp} % provide euro and other symbols
\else % if luatex or xetex
  \usepackage{unicode-math}
  \defaultfontfeatures{Scale=MatchLowercase}
  \defaultfontfeatures[\rmfamily]{Ligatures=TeX,Scale=1}
\fi
% Use upquote if available, for straight quotes in verbatim environments
\IfFileExists{upquote.sty}{\usepackage{upquote}}{}
\IfFileExists{microtype.sty}{% use microtype if available
  \usepackage[]{microtype}
  \UseMicrotypeSet[protrusion]{basicmath} % disable protrusion for tt fonts
}{}
\makeatletter
\@ifundefined{KOMAClassName}{% if non-KOMA class
  \IfFileExists{parskip.sty}{%
    \usepackage{parskip}
  }{% else
    \setlength{\parindent}{0pt}
    \setlength{\parskip}{6pt plus 2pt minus 1pt}}
}{% if KOMA class
  \KOMAoptions{parskip=half}}
\makeatother
\usepackage{xcolor}
\usepackage[margin=1in]{geometry}
\usepackage{color}
\usepackage{fancyvrb}
\newcommand{\VerbBar}{|}
\newcommand{\VERB}{\Verb[commandchars=\\\{\}]}
\DefineVerbatimEnvironment{Highlighting}{Verbatim}{commandchars=\\\{\}}
% Add ',fontsize=\small' for more characters per line
\usepackage{framed}
\definecolor{shadecolor}{RGB}{248,248,248}
\newenvironment{Shaded}{\begin{snugshade}}{\end{snugshade}}
\newcommand{\AlertTok}[1]{\textcolor[rgb]{0.94,0.16,0.16}{#1}}
\newcommand{\AnnotationTok}[1]{\textcolor[rgb]{0.56,0.35,0.01}{\textbf{\textit{#1}}}}
\newcommand{\AttributeTok}[1]{\textcolor[rgb]{0.77,0.63,0.00}{#1}}
\newcommand{\BaseNTok}[1]{\textcolor[rgb]{0.00,0.00,0.81}{#1}}
\newcommand{\BuiltInTok}[1]{#1}
\newcommand{\CharTok}[1]{\textcolor[rgb]{0.31,0.60,0.02}{#1}}
\newcommand{\CommentTok}[1]{\textcolor[rgb]{0.56,0.35,0.01}{\textit{#1}}}
\newcommand{\CommentVarTok}[1]{\textcolor[rgb]{0.56,0.35,0.01}{\textbf{\textit{#1}}}}
\newcommand{\ConstantTok}[1]{\textcolor[rgb]{0.00,0.00,0.00}{#1}}
\newcommand{\ControlFlowTok}[1]{\textcolor[rgb]{0.13,0.29,0.53}{\textbf{#1}}}
\newcommand{\DataTypeTok}[1]{\textcolor[rgb]{0.13,0.29,0.53}{#1}}
\newcommand{\DecValTok}[1]{\textcolor[rgb]{0.00,0.00,0.81}{#1}}
\newcommand{\DocumentationTok}[1]{\textcolor[rgb]{0.56,0.35,0.01}{\textbf{\textit{#1}}}}
\newcommand{\ErrorTok}[1]{\textcolor[rgb]{0.64,0.00,0.00}{\textbf{#1}}}
\newcommand{\ExtensionTok}[1]{#1}
\newcommand{\FloatTok}[1]{\textcolor[rgb]{0.00,0.00,0.81}{#1}}
\newcommand{\FunctionTok}[1]{\textcolor[rgb]{0.00,0.00,0.00}{#1}}
\newcommand{\ImportTok}[1]{#1}
\newcommand{\InformationTok}[1]{\textcolor[rgb]{0.56,0.35,0.01}{\textbf{\textit{#1}}}}
\newcommand{\KeywordTok}[1]{\textcolor[rgb]{0.13,0.29,0.53}{\textbf{#1}}}
\newcommand{\NormalTok}[1]{#1}
\newcommand{\OperatorTok}[1]{\textcolor[rgb]{0.81,0.36,0.00}{\textbf{#1}}}
\newcommand{\OtherTok}[1]{\textcolor[rgb]{0.56,0.35,0.01}{#1}}
\newcommand{\PreprocessorTok}[1]{\textcolor[rgb]{0.56,0.35,0.01}{\textit{#1}}}
\newcommand{\RegionMarkerTok}[1]{#1}
\newcommand{\SpecialCharTok}[1]{\textcolor[rgb]{0.00,0.00,0.00}{#1}}
\newcommand{\SpecialStringTok}[1]{\textcolor[rgb]{0.31,0.60,0.02}{#1}}
\newcommand{\StringTok}[1]{\textcolor[rgb]{0.31,0.60,0.02}{#1}}
\newcommand{\VariableTok}[1]{\textcolor[rgb]{0.00,0.00,0.00}{#1}}
\newcommand{\VerbatimStringTok}[1]{\textcolor[rgb]{0.31,0.60,0.02}{#1}}
\newcommand{\WarningTok}[1]{\textcolor[rgb]{0.56,0.35,0.01}{\textbf{\textit{#1}}}}
\usepackage{longtable,booktabs,array}
\usepackage{calc} % for calculating minipage widths
% Correct order of tables after \paragraph or \subparagraph
\usepackage{etoolbox}
\makeatletter
\patchcmd\longtable{\par}{\if@noskipsec\mbox{}\fi\par}{}{}
\makeatother
% Allow footnotes in longtable head/foot
\IfFileExists{footnotehyper.sty}{\usepackage{footnotehyper}}{\usepackage{footnote}}
\makesavenoteenv{longtable}
\usepackage{graphicx}
\makeatletter
\def\maxwidth{\ifdim\Gin@nat@width>\linewidth\linewidth\else\Gin@nat@width\fi}
\def\maxheight{\ifdim\Gin@nat@height>\textheight\textheight\else\Gin@nat@height\fi}
\makeatother
% Scale images if necessary, so that they will not overflow the page
% margins by default, and it is still possible to overwrite the defaults
% using explicit options in \includegraphics[width, height, ...]{}
\setkeys{Gin}{width=\maxwidth,height=\maxheight,keepaspectratio}
% Set default figure placement to htbp
\makeatletter
\def\fps@figure{htbp}
\makeatother
\setlength{\emergencystretch}{3em} % prevent overfull lines
\providecommand{\tightlist}{%
  \setlength{\itemsep}{0pt}\setlength{\parskip}{0pt}}
\setcounter{secnumdepth}{-\maxdimen} % remove section numbering
\ifLuaTeX
  \usepackage{selnolig}  % disable illegal ligatures
\fi
\IfFileExists{bookmark.sty}{\usepackage{bookmark}}{\usepackage{hyperref}}
\IfFileExists{xurl.sty}{\usepackage{xurl}}{} % add URL line breaks if available
\urlstyle{same} % disable monospaced font for URLs
\hypersetup{
  pdftitle={projet\_R \_2023},
  pdfauthor={Dieynaba},
  hidelinks,
  pdfcreator={LaTeX via pandoc}}

\title{projet\_R \_2023}
\author{Dieynaba}
\date{2023-07-20}

\begin{document}
\maketitle

\hypertarget{partie-1}{%
\subsection{1 partie 1}\label{partie-1}}

L'objectif de ce projet est que nous appliquons les outils que nous
avons étudiés dans le cours du logiciel statistique R, dans le cas d'une
étude de cas réelle.

\begin{Shaded}
\begin{Highlighting}[]
\CommentTok{\#les packages essentiels pour le projet}
\FunctionTok{library}\NormalTok{(}\StringTok{"haven"}\NormalTok{)}
\FunctionTok{library}\NormalTok{(}\StringTok{"readxl"}\NormalTok{)}
\FunctionTok{library}\NormalTok{(}\StringTok{"janitor"}\NormalTok{)}
\FunctionTok{library}\NormalTok{(}\StringTok{"gtsummary"}\NormalTok{)}
\FunctionTok{library}\NormalTok{(}\StringTok{"dplyr"}\NormalTok{)}
\FunctionTok{library}\NormalTok{(}\StringTok{"knitr"}\NormalTok{)}
\FunctionTok{library}\NormalTok{(ggplot2)}
\FunctionTok{library}\NormalTok{(sf)}
\FunctionTok{library}\NormalTok{(st)}
\end{Highlighting}
\end{Shaded}

\hypertarget{importation-et-mise-en-forme}{%
\subsection{1.2 Importation et mise en
forme}\label{importation-et-mise-en-forme}}

\begin{Shaded}
\begin{Highlighting}[]
\NormalTok{Base\_partie1}\OtherTok{\textless{}{-}} \FunctionTok{read\_excel}\NormalTok{(}\StringTok{"Base\_Partie 1.xlsx"}\NormalTok{) }\CommentTok{\#on importe la base}
\NormalTok{projet}\OtherTok{\textless{}{-}}\FunctionTok{data.frame}\NormalTok{(Base\_partie1)}\CommentTok{\#création du dataframe}
\NormalTok{base1}\OtherTok{\textless{}{-}}\NormalTok{projet}\SpecialCharTok{|\textgreater{}}\NormalTok{ dplyr }\SpecialCharTok{::} \FunctionTok{select}\NormalTok{(}\SpecialCharTok{{-}}\FunctionTok{c}\NormalTok{(}\StringTok{"key"}\NormalTok{))}\CommentTok{\#selection de toutes les variables de la base sauf la variable key}
\end{Highlighting}
\end{Shaded}

\#\#Tableau qui résume les valeurs manquantes par variables

\begin{Shaded}
\begin{Highlighting}[]
\NormalTok{val\_manquantes}\OtherTok{\textless{}{-}}\FunctionTok{data.frame}\NormalTok{(}\AttributeTok{nbr\_valmanquantes=}\FunctionTok{colSums}\NormalTok{(}\FunctionTok{is.na}\NormalTok{(projet)),}\AttributeTok{frequency=}\NormalTok{(}\FunctionTok{colSums}\NormalTok{(}\FunctionTok{is.na}\NormalTok{(projet)}\SpecialCharTok{/}\FunctionTok{nrow}\NormalTok{(projet))}\SpecialCharTok{*}\DecValTok{100}\NormalTok{))}
\FunctionTok{kable}\NormalTok{(val\_manquantes)}\CommentTok{\# création du tableau qui résume les valeurs manquantes par variables}
\end{Highlighting}
\end{Shaded}

\begin{longtable}[]{@{}lrr@{}}
\toprule()
& nbr\_valmanquantes & frequency \\
\midrule()
\endhead
key & 0 & 0.0 \\
q1 & 0 & 0.0 \\
q2 & 0 & 0.0 \\
q23 & 0 & 0.0 \\
q24 & 0 & 0.0 \\
q24a\_1 & 0 & 0.0 \\
q24a\_2 & 0 & 0.0 \\
q24a\_3 & 0 & 0.0 \\
q24a\_4 & 0 & 0.0 \\
q24a\_5 & 0 & 0.0 \\
q24a\_6 & 0 & 0.0 \\
q24a\_7 & 0 & 0.0 \\
q24a\_9 & 0 & 0.0 \\
q24a\_10 & 0 & 0.0 \\
q25 & 0 & 0.0 \\
q26 & 0 & 0.0 \\
q12 & 0 & 0.0 \\
q14b & 1 & 0.4 \\
q16 & 1 & 0.4 \\
q17 & 131 & 52.4 \\
q19 & 120 & 48.0 \\
q20 & 0 & 0.0 \\
filiere\_1 & 0 & 0.0 \\
filiere\_2 & 0 & 0.0 \\
filiere\_3 & 0 & 0.0 \\
filiere\_4 & 0 & 0.0 \\
q8 & 0 & 0.0 \\
q81 & 0 & 0.0 \\
gps\_menlatitude & 0 & 0.0 \\
gps\_menlongitude & 0 & 0.0 \\
submissiondate & 0 & 0.0 \\
start & 0 & 0.0 \\
today & 0 & 0.0 \\
\bottomrule()
\end{longtable}

Au regard du tableau ci\_dessus on voit que la variable key ne contient
aucune de valeur manquante.

\hypertarget{cruxe9ation-des-variables}{%
\subsection{1.3 Création des
variables}\label{cruxe9ation-des-variables}}

Avec la fonction rename(), on renomme nos trois variables que l'on met
dans un nouveau dataframe, puis on utilise la fonction mutate() pour
créer la variable sexe.La création du dataframe Langues se fait
facilement en appliquant la fonction select() à notre base.

\begin{Shaded}
\begin{Highlighting}[]
\NormalTok{newdata}\OtherTok{\textless{}{-}}\NormalTok{projet}\SpecialCharTok{\%\textgreater{}\%} \CommentTok{\#création d\textquotesingle{}un nouveau dataframe où les 3variables seront renommés}
  \FunctionTok{rename}\NormalTok{(}\AttributeTok{region=}\NormalTok{q1,}\AttributeTok{departement=}\NormalTok{q2,}\AttributeTok{sexe=}\NormalTok{q23)}
\NormalTok{newdata}\OtherTok{\textless{}{-}}\NormalTok{newdata}\SpecialCharTok{\%\textgreater{}\%} \CommentTok{\# création de la variable sexe2 en conservant notre base initiale}
  \FunctionTok{mutate}\NormalTok{(}\AttributeTok{sexe\_2=}\FunctionTok{ifelse}\NormalTok{(sexe}\SpecialCharTok{==}\StringTok{"Femme"}\NormalTok{,}\DecValTok{1}\NormalTok{,}\DecValTok{0}\NormalTok{))}
\NormalTok{Langues}\OtherTok{\textless{}{-}}\FunctionTok{select}\NormalTok{(newdata,}\StringTok{\textquotesingle{}key\textquotesingle{}}\NormalTok{,}\FunctionTok{starts\_with}\NormalTok{(}\StringTok{\textquotesingle{}q24a\textquotesingle{}}\NormalTok{))}\CommentTok{\#création du dataframe Langues}
\end{Highlighting}
\end{Shaded}

pour créer la variable parle

\hypertarget{statistique-descriptive}{%
\subsection{Statistique descriptive}\label{statistique-descriptive}}

faisons un statistique descriptive sur les variables de notre choix.
Dans cette partie, il s'agira de voir la répartion des PME suivant la
region et le sexe du dirigeant de la PME. le tableau ci\_dessous montre
cette répartition.

\begin{Shaded}
\begin{Highlighting}[]
\NormalTok{tab}\OtherTok{\textless{}{-}}\FunctionTok{table}\NormalTok{(projet}\SpecialCharTok{$}\NormalTok{q1,projet}\SpecialCharTok{$}\NormalTok{q23)}
\NormalTok{knitr}\SpecialCharTok{::}\FunctionTok{kable}\NormalTok{(tab,}\AttributeTok{caption=}\StringTok{"Répartion des PME suivant le sexe et la région"}\NormalTok{,}\AttributeTok{escape=}\NormalTok{F)}
\end{Highlighting}
\end{Shaded}

\begin{longtable}[]{@{}lrr@{}}
\caption{Répartion des PME suivant le sexe et la région}\tabularnewline
\toprule()
& Femme & Homme \\
\midrule()
\endfirsthead
\toprule()
& Femme & Homme \\
\midrule()
\endhead
Dakar & 0 & 1 \\
Diourbel & 28 & 6 \\
Fatick & 23 & 7 \\
Kaffrine & 8 & 0 \\
Kaolack & 20 & 1 \\
Kolda & 4 & 5 \\
Saint-Louis & 22 & 20 \\
Sédhiou & 1 & 3 \\
Thiès & 48 & 3 \\
Ziguinchor & 37 & 13 \\
\bottomrule()
\end{longtable}

Dans ce qui suit, il s'agit de voir la répartition des PME suivant
l'activité principale et région. Le tableau ci-dessous montre cette
répartition.

\begin{Shaded}
\begin{Highlighting}[]
\NormalTok{tab}\OtherTok{\textless{}{-}}\FunctionTok{table}\NormalTok{(projet}\SpecialCharTok{$}\NormalTok{q1,projet}\SpecialCharTok{$}\NormalTok{q8)}
\NormalTok{knitr}\SpecialCharTok{::}\FunctionTok{kable}\NormalTok{(tab,}\AttributeTok{caption=}\StringTok{"Répartion des PME suivant l\textquotesingle{}activité principale et la région et la région"}\NormalTok{,}\AttributeTok{escape=}\NormalTok{F)}
\end{Highlighting}
\end{Shaded}

\begin{longtable}[]{@{}
  >{\raggedright\arraybackslash}p{(\columnwidth - 20\tabcolsep) * \real{0.0393}}
  >{\raggedleft\arraybackslash}p{(\columnwidth - 20\tabcolsep) * \real{0.0197}}
  >{\raggedleft\arraybackslash}p{(\columnwidth - 20\tabcolsep) * \real{0.0557}}
  >{\raggedleft\arraybackslash}p{(\columnwidth - 20\tabcolsep) * \real{0.1049}}
  >{\raggedleft\arraybackslash}p{(\columnwidth - 20\tabcolsep) * \real{0.1377}}
  >{\raggedleft\arraybackslash}p{(\columnwidth - 20\tabcolsep) * \real{0.1443}}
  >{\raggedleft\arraybackslash}p{(\columnwidth - 20\tabcolsep) * \real{0.0951}}
  >{\raggedleft\arraybackslash}p{(\columnwidth - 20\tabcolsep) * \real{0.0918}}
  >{\raggedleft\arraybackslash}p{(\columnwidth - 20\tabcolsep) * \real{0.1180}}
  >{\raggedleft\arraybackslash}p{(\columnwidth - 20\tabcolsep) * \real{0.1213}}
  >{\raggedleft\arraybackslash}p{(\columnwidth - 20\tabcolsep) * \real{0.0721}}@{}}
\caption{Répartion des PME suivant l'activité principale et la région et
la région}\tabularnewline
\toprule()
\begin{minipage}[b]{\linewidth}\raggedright
\end{minipage} & \begin{minipage}[b]{\linewidth}\raggedleft
Aucun
\end{minipage} & \begin{minipage}[b]{\linewidth}\raggedleft
Autre a preciser
\end{minipage} & \begin{minipage}[b]{\linewidth}\raggedleft
Tansformation d'autres céréales
\end{minipage} & \begin{minipage}[b]{\linewidth}\raggedleft
Transformation d'autres fruits et legumes
\end{minipage} & \begin{minipage}[b]{\linewidth}\raggedleft
Transformation d'autres produits oléagineux
\end{minipage} & \begin{minipage}[b]{\linewidth}\raggedleft
Transformation de l'arachide
\end{minipage} & \begin{minipage}[b]{\linewidth}\raggedleft
Transformation de la mangue
\end{minipage} & \begin{minipage}[b]{\linewidth}\raggedleft
Transformation de la noix de cajoux
\end{minipage} & \begin{minipage}[b]{\linewidth}\raggedleft
Transformation de la pomme de cajoux
\end{minipage} & \begin{minipage}[b]{\linewidth}\raggedleft
Transformation du riz
\end{minipage} \\
\midrule()
\endfirsthead
\toprule()
\begin{minipage}[b]{\linewidth}\raggedright
\end{minipage} & \begin{minipage}[b]{\linewidth}\raggedleft
Aucun
\end{minipage} & \begin{minipage}[b]{\linewidth}\raggedleft
Autre a preciser
\end{minipage} & \begin{minipage}[b]{\linewidth}\raggedleft
Tansformation d'autres céréales
\end{minipage} & \begin{minipage}[b]{\linewidth}\raggedleft
Transformation d'autres fruits et legumes
\end{minipage} & \begin{minipage}[b]{\linewidth}\raggedleft
Transformation d'autres produits oléagineux
\end{minipage} & \begin{minipage}[b]{\linewidth}\raggedleft
Transformation de l'arachide
\end{minipage} & \begin{minipage}[b]{\linewidth}\raggedleft
Transformation de la mangue
\end{minipage} & \begin{minipage}[b]{\linewidth}\raggedleft
Transformation de la noix de cajoux
\end{minipage} & \begin{minipage}[b]{\linewidth}\raggedleft
Transformation de la pomme de cajoux
\end{minipage} & \begin{minipage}[b]{\linewidth}\raggedleft
Transformation du riz
\end{minipage} \\
\midrule()
\endhead
Dakar & 0 & 0 & 0 & 0 & 0 & 0 & 0 & 1 & 0 & 0 \\
Diourbel & 5 & 0 & 3 & 0 & 0 & 26 & 0 & 0 & 0 & 0 \\
Fatick & 0 & 2 & 3 & 4 & 0 & 7 & 0 & 14 & 0 & 0 \\
Kaffrine & 0 & 0 & 8 & 0 & 0 & 0 & 0 & 0 & 0 & 0 \\
Kaolack & 0 & 0 & 14 & 0 & 1 & 6 & 0 & 0 & 0 & 0 \\
Kolda & 0 & 0 & 0 & 0 & 0 & 1 & 3 & 5 & 0 & 0 \\
Saint-Louis & 0 & 0 & 1 & 0 & 0 & 0 & 0 & 0 & 0 & 41 \\
Sédhiou & 0 & 0 & 0 & 1 & 0 & 0 & 0 & 2 & 1 & 0 \\
Thiès & 0 & 0 & 28 & 9 & 0 & 7 & 4 & 0 & 0 & 3 \\
Ziguinchor & 0 & 2 & 0 & 0 & 0 & 0 & 28 & 10 & 8 & 2 \\
\bottomrule()
\end{longtable}

Dans cette partie, nous allons voir la répartiotion des PME suivant
l'age et le sexe du responsable de la PME. Mais avant celà, il est
necessaire de regrouper la variable age en tranche d'age pour faciliter
l'analyse.

\begin{Shaded}
\begin{Highlighting}[]
\NormalTok{Age}\OtherTok{\textless{}{-}}\NormalTok{projet}\SpecialCharTok{$}\NormalTok{q24}\CommentTok{\# on récupere la varible qui contient l\textquotesingle{}age des dirigeants des PME}
\NormalTok{points\_de\_coupe}\OtherTok{\textless{}{-}}\FunctionTok{c}\NormalTok{(}\DecValTok{0}\NormalTok{,}\DecValTok{30}\NormalTok{,}\DecValTok{40}\NormalTok{,}\DecValTok{50}\NormalTok{,}\DecValTok{60}\NormalTok{,}\ConstantTok{Inf}\NormalTok{)}\CommentTok{\# on définit les points de coupure }
\NormalTok{tranche\_age}\OtherTok{\textless{}{-}}\FunctionTok{cut}\NormalTok{(Age,}\AttributeTok{breaks=}\NormalTok{points\_de\_coupe,}\AttributeTok{labels=}\FunctionTok{c}\NormalTok{(}\StringTok{"0{-}30"}\NormalTok{,}\StringTok{"31{-}40"}\NormalTok{,}\StringTok{"41{-}50"}\NormalTok{,}\StringTok{"51{-}60"}\NormalTok{,}\StringTok{"61+"}\NormalTok{))}\CommentTok{\#on met la variable age sous forme de tranche d\textquotesingle{}age}
\NormalTok{tab}\OtherTok{\textless{}{-}}\FunctionTok{table}\NormalTok{(projet}\SpecialCharTok{$}\NormalTok{q23,tranche\_age)}\CommentTok{\# puis on fait le croisement}
\NormalTok{knitr}\SpecialCharTok{::}\FunctionTok{kable}\NormalTok{(tab,}\AttributeTok{caption=}\StringTok{"Répartion des PME suivant le sexe et l\textquotesingle{}age "}\NormalTok{,}\AttributeTok{escape=}\NormalTok{F)}
\end{Highlighting}
\end{Shaded}

\begin{longtable}[]{@{}lrrrrr@{}}
\caption{Répartion des PME suivant le sexe et l'age}\tabularnewline
\toprule()
& 0-30 & 31-40 & 41-50 & 51-60 & 61+ \\
\midrule()
\endfirsthead
\toprule()
& 0-30 & 31-40 & 41-50 & 51-60 & 61+ \\
\midrule()
\endhead
Femme & 3 & 20 & 42 & 59 & 67 \\
Homme & 3 & 16 & 12 & 16 & 12 \\
\bottomrule()
\end{longtable}

Dans cette section , il s'agira tout simplement de voir la répartition
des PME suivant le nombre d'annees d'expérience et le niveau
d'instruction.

\begin{Shaded}
\begin{Highlighting}[]
\NormalTok{Nbr\_annee}\OtherTok{\textless{}{-}}\NormalTok{projet}\SpecialCharTok{$}\NormalTok{q26}
\NormalTok{points\_de\_coupe}\OtherTok{\textless{}{-}}\FunctionTok{c}\NormalTok{(}\DecValTok{0}\NormalTok{,}\DecValTok{5}\NormalTok{,}\DecValTok{15}\NormalTok{,}\DecValTok{25}\NormalTok{,}\DecValTok{35}\NormalTok{,}\ConstantTok{Inf}\NormalTok{)}
\NormalTok{tranche}\OtherTok{\textless{}{-}}\FunctionTok{cut}\NormalTok{(Nbr\_annee,}\AttributeTok{breaks=}\NormalTok{points\_de\_coupe,}\AttributeTok{labels=}\FunctionTok{c}\NormalTok{(}\StringTok{"0{-}5"}\NormalTok{,}\StringTok{"6{-}15"}\NormalTok{,}\StringTok{"16{-}25"}\NormalTok{,}\StringTok{"26{-}35"}\NormalTok{,}\StringTok{"36+"}\NormalTok{))}
\NormalTok{tab}\OtherTok{\textless{}{-}}\FunctionTok{table}\NormalTok{(projet}\SpecialCharTok{$}\NormalTok{q25,tranche)}
\NormalTok{knitr}\SpecialCharTok{::}\FunctionTok{kable}\NormalTok{(tab,}\AttributeTok{caption=}\StringTok{"Répartion des PME suivant le niveau d\textquotesingle{}instruction et le nombre d\textquotesingle{}années d\textquotesingle{}expérience "}\NormalTok{,}\AttributeTok{escape=}\NormalTok{F)}
\end{Highlighting}
\end{Shaded}

\begin{longtable}[]{@{}lrrrrr@{}}
\caption{Répartion des PME suivant le niveau d'instruction et le nombre
d'années d'expérience}\tabularnewline
\toprule()
& 0-5 & 6-15 & 16-25 & 26-35 & 36+ \\
\midrule()
\endfirsthead
\toprule()
& 0-5 & 6-15 & 16-25 & 26-35 & 36+ \\
\midrule()
\endhead
Aucun niveau & 11 & 32 & 25 & 9 & 2 \\
Niveau primaire & 16 & 20 & 10 & 8 & 2 \\
Niveau secondaire & 15 & 33 & 17 & 6 & 3 \\
Niveau Superieur & 20 & 13 & 7 & 0 & 1 \\
\bottomrule()
\end{longtable}

\hypertarget{partie-2}{%
\subsection{Partie 2}\label{partie-2}}

\hypertarget{netoyage-et-gestion-des-donnuxe9es}{%
\subsection{Netoyage et gestion des
données}\label{netoyage-et-gestion-des-donnuxe9es}}

On importe d'abord la base avant de procéder au netoyage des données.

\begin{Shaded}
\begin{Highlighting}[]
\NormalTok{Base\_partie2}\OtherTok{\textless{}{-}} \FunctionTok{read\_excel}\NormalTok{(}\StringTok{"Base\_Partie 2.xlsx"}\NormalTok{)}\CommentTok{\# on importe d\textquotesingle{}abord la base}
\NormalTok{projet2}\OtherTok{\textless{}{-}}\FunctionTok{data.frame}\NormalTok{(Base\_partie2)}\CommentTok{\# création du dataframe projet2}
\end{Highlighting}
\end{Shaded}

• Renommons la variable ``country\_destination'' en ``destination''et
définissons les valeurs négatives comme manquantes.

\begin{Shaded}
\begin{Highlighting}[]
\NormalTok{projet2}\OtherTok{\textless{}{-}}\NormalTok{projet2}\SpecialCharTok{\%\textgreater{}\%} 
  \FunctionTok{rename}\NormalTok{(}\AttributeTok{destination=}\StringTok{"country\_destination"}\NormalTok{)}\CommentTok{\# on renomme la country\_destination en destination}
\NormalTok{projet2}\SpecialCharTok{$}\NormalTok{destination}\OtherTok{\textless{}{-}}\FunctionTok{ifelse}\NormalTok{(projet2}\SpecialCharTok{$}\NormalTok{destination}\SpecialCharTok{\textless{}}\DecValTok{0}\NormalTok{,}\ConstantTok{NA}\NormalTok{,projet2}\SpecialCharTok{$}\NormalTok{destination)}\CommentTok{\# on définit les valeurs négatives commes manquante}
\end{Highlighting}
\end{Shaded}

• Créer une nouvelle variable contenant des tranches d'âge de 5 ans en
utilisant la variable ``age''.

\begin{Shaded}
\begin{Highlighting}[]
\NormalTok{Var}\OtherTok{\textless{}{-}}\NormalTok{projet2}\SpecialCharTok{$}\NormalTok{age}\CommentTok{\# on récupere la variable age dans la base projet2}
\NormalTok{points\_de\_coupe}\OtherTok{\textless{}{-}}\FunctionTok{c}\NormalTok{(}\DecValTok{5}\NormalTok{,}\DecValTok{10}\NormalTok{,}\DecValTok{15}\NormalTok{,}\DecValTok{20}\NormalTok{,}\DecValTok{25}\NormalTok{,}\DecValTok{30}\NormalTok{,}\DecValTok{35}\NormalTok{,}\DecValTok{40}\NormalTok{,}\ConstantTok{Inf}\NormalTok{)}\CommentTok{\# definition des extremités de nos tranches d\textquotesingle{}age}
\NormalTok{Ag}\OtherTok{\textless{}{-}}\FunctionTok{cut}\NormalTok{(Var,}\AttributeTok{breaks=}\NormalTok{points\_de\_coupe,}\AttributeTok{labels=}\FunctionTok{c}\NormalTok{(}\StringTok{"5{-}10"}\NormalTok{,}\StringTok{"11{-}15"}\NormalTok{,}\StringTok{"16{-}20"}\NormalTok{,}\StringTok{"21{-}25"}\NormalTok{,}\StringTok{"26{-}30"}\NormalTok{,}\StringTok{"31{-}35"}\NormalTok{,}\StringTok{"36{-}40"}\NormalTok{,}\StringTok{"41+"}\NormalTok{))}\CommentTok{\# creation de la variable qui recupere les tranches d\textquotesingle{}age}
\NormalTok{projet2}\OtherTok{\textless{}{-}}\NormalTok{projet2}\SpecialCharTok{\%\textgreater{}\%} \CommentTok{\# creation de la variable Age au niveau de projet2 en utilisant la variable Ag}
  \FunctionTok{mutate}\NormalTok{(}\AttributeTok{Age=}\NormalTok{Ag)}
\end{Highlighting}
\end{Shaded}

• Créons une nouvelle variable contenant le nombre d'entretiens réalisés
par chaque agent recenseur.

\hypertarget{un-peu-de-cartographie}{%
\subsection{3 Un peu de cartographie}\label{un-peu-de-cartographie}}

\#\#3.1Transformation du dataframe en données géographiques Avant de
tranformer notre data frame en données geographiques, on télécharge
d'abord les données du sénégal sur le site gadm avec le format shapefile
qui contient les données spatiales.

\begin{Shaded}
\begin{Highlighting}[]
\NormalTok{senegal }\OtherTok{\textless{}{-}} \FunctionTok{st\_read}\NormalTok{(}\StringTok{"gadm41\_SEN\_shp/gadm41\_SEN\_1.shp"}\NormalTok{)}\CommentTok{\# pour lire les fichiers spatiales}
\end{Highlighting}
\end{Shaded}

\begin{verbatim}
## Reading layer `gadm41_SEN_1' from data source 
##   `C:\Users\user\Documents\Projet_R_2023_KA_Dieynaba\gadm41_SEN_shp\gadm41_SEN_1.shp' 
##   using driver `ESRI Shapefile'
## Simple feature collection with 14 features and 11 fields
## Geometry type: MULTIPOLYGON
## Dimension:     XY
## Bounding box:  xmin: -17.54319 ymin: 12.30786 xmax: -11.34247 ymax: 16.69207
## Geodetic CRS:  WGS 84
\end{verbatim}

\begin{Shaded}
\begin{Highlighting}[]
\NormalTok{projet\_map}\OtherTok{\textless{}{-}}\FunctionTok{st\_as\_sf}\NormalTok{(projet,}\AttributeTok{coords=}\FunctionTok{c}\NormalTok{(}\StringTok{"gps\_menlongitude"}\NormalTok{,}\StringTok{"gps\_menlatitude"}\NormalTok{), }\AttributeTok{crs =}\FunctionTok{st\_crs}\NormalTok{(senegal))}\CommentTok{\# tranformation du dataframe en données géographiques}
\NormalTok{projet\_map}\OtherTok{\textless{}{-}} \FunctionTok{st\_join}\NormalTok{(projet\_map, senegal)}\CommentTok{\# on joint les deux}
\end{Highlighting}
\end{Shaded}

\#\#3.2 Représentation spatiales des PME suivant le sexe Pour faire la
représentation spatiale, on utise la fonction ggplot() qui est dans le
package ggplot2.

\begin{Shaded}
\begin{Highlighting}[]
\FunctionTok{ggplot}\NormalTok{(newdata) }\SpecialCharTok{+}\CommentTok{\# on vas chercher les variables dans la base newdata,puis}
  \FunctionTok{geom\_sf}\NormalTok{(}\AttributeTok{data=}\NormalTok{senegal) }\SpecialCharTok{+}\CommentTok{\# faire la représentation à partir de la carte du senegal}
  \FunctionTok{geom\_point}\NormalTok{(}\FunctionTok{aes}\NormalTok{(}\AttributeTok{x =}\NormalTok{ gps\_menlongitude,}\AttributeTok{y=}\NormalTok{gps\_menlatitude, }\AttributeTok{color=}\NormalTok{sexe))}\SpecialCharTok{+}\CommentTok{\#on fait la représentation suivant la latitude et la longitude et on colorie suivant le sexe}
  \FunctionTok{labs}\NormalTok{(}\AttributeTok{title =} \StringTok{"Répartition spatiale des PME suivant le sexe"}\NormalTok{,}\AttributeTok{x =} \StringTok{"gps\_menlongitude"}\NormalTok{,}\AttributeTok{y=}\StringTok{"gps\_menlatitude"}\NormalTok{)}\SpecialCharTok{+}\CommentTok{\# titre de notre carte}
  \FunctionTok{theme\_minimal}\NormalTok{()}
\end{Highlighting}
\end{Shaded}

\includegraphics{rmarkdown_files/figure-latex/unnamed-chunk-15-1.pdf}

\#\#3.3 Représentation spatiales des PME suivant le niveau d'instruction
Renommons d'abord la variable q25 en Niveau d'instruction.

\begin{Shaded}
\begin{Highlighting}[]
\NormalTok{newdata}\OtherTok{\textless{}{-}}\NormalTok{newdata}\SpecialCharTok{\%\textgreater{}\%}
  \FunctionTok{rename}\NormalTok{(}\AttributeTok{Niveau\_instruction=}\NormalTok{q25)}
\end{Highlighting}
\end{Shaded}

On applique le même procédé que précédemment sauf que à la place de la
variable sexe on a la variable Niveau\_instruction.

\begin{Shaded}
\begin{Highlighting}[]
\FunctionTok{ggplot}\NormalTok{(newdata)}\SpecialCharTok{+}
  \FunctionTok{geom\_sf}\NormalTok{(}\AttributeTok{data=}\NormalTok{senegal) }\SpecialCharTok{+}
  \FunctionTok{geom\_point}\NormalTok{(}\FunctionTok{aes}\NormalTok{(}\AttributeTok{x =}\NormalTok{ gps\_menlongitude,}\AttributeTok{y=}\NormalTok{gps\_menlatitude, }\AttributeTok{color=}\NormalTok{Niveau\_instruction))}\SpecialCharTok{+}
  \FunctionTok{labs}\NormalTok{(}\AttributeTok{title =} \StringTok{"Répartition spatiale des PME suivant le niveau d\textquotesingle{}instruction"}\NormalTok{,}\AttributeTok{x =} \StringTok{"gps\_menlongitude"}\NormalTok{,}\AttributeTok{y=}\StringTok{"gps\_menlatitude"}\NormalTok{)}\SpecialCharTok{+}
  \FunctionTok{theme\_minimal}\NormalTok{()}
\end{Highlighting}
\end{Shaded}

\includegraphics{rmarkdown_files/figure-latex/unnamed-chunk-17-1.pdf}

\#\#3.4 Faisons une analyse spatiale de notre choix La répartition
spatiale des PME suivant le sexe montre que les PME sont majoritairement
dirigés par des femmes et ce phénoméne est plus visible dans les Régions
de THIES, DIOURBEl,et ZIGUINCHOR, par contre dans la région de
SAINT\_LOUIS on note une répatition presque identique suivant les deux
sexes. Il ressort également des résultats que les PME dans les régions
de KAFFRINE et KAOLACK sont uniquement dirigés par les femmes. La
répartition spatiale des PME suivant le niveau d'instruction montre que
les PME dont le dirigeants n'a pas fait détude sont plus visibles dans
les régions de THIES et DIOURBEl,par celles dont les dirigeants a un
niveau supérieur sont plus visibles dans les régions de
DAKAR,SAINT\_LOUIS et ZIGUINCHOR. Il ressort également des résultats que
dans la région de KOLDA la majorité des PME le dirigeant n'a fait que
des études primaires.

\end{document}
